\documentclass[]{article}
\usepackage{lmodern}
\usepackage{amssymb,amsmath}
\usepackage{ifxetex,ifluatex}
\usepackage{fixltx2e} % provides \textsubscript
\ifnum 0\ifxetex 1\fi\ifluatex 1\fi=0 % if pdftex
  \usepackage[T1]{fontenc}
  \usepackage[utf8]{inputenc}
\else % if luatex or xelatex
  \ifxetex
    \usepackage{mathspec}
  \else
    \usepackage{fontspec}
  \fi
  \defaultfontfeatures{Ligatures=TeX,Scale=MatchLowercase}
\fi
% use upquote if available, for straight quotes in verbatim environments
\IfFileExists{upquote.sty}{\usepackage{upquote}}{}
% use microtype if available
\IfFileExists{microtype.sty}{%
\usepackage[]{microtype}
\UseMicrotypeSet[protrusion]{basicmath} % disable protrusion for tt fonts
}{}
\PassOptionsToPackage{hyphens}{url} % url is loaded by hyperref
\usepackage[unicode=true]{hyperref}
\hypersetup{
            pdftitle={Homework - DSSC Core Course},
            pdfauthor={Aleksandra Greshnova},
            pdfborder={0 0 0},
            breaklinks=true}
\urlstyle{same}  % don't use monospace font for urls
\usepackage[margin=1in]{geometry}
\usepackage{color}
\usepackage{fancyvrb}
\newcommand{\VerbBar}{|}
\newcommand{\VERB}{\Verb[commandchars=\\\{\}]}
\DefineVerbatimEnvironment{Highlighting}{Verbatim}{commandchars=\\\{\}}
% Add ',fontsize=\small' for more characters per line
\usepackage{framed}
\definecolor{shadecolor}{RGB}{248,248,248}
\newenvironment{Shaded}{\begin{snugshade}}{\end{snugshade}}
\newcommand{\KeywordTok}[1]{\textcolor[rgb]{0.13,0.29,0.53}{\textbf{#1}}}
\newcommand{\DataTypeTok}[1]{\textcolor[rgb]{0.13,0.29,0.53}{#1}}
\newcommand{\DecValTok}[1]{\textcolor[rgb]{0.00,0.00,0.81}{#1}}
\newcommand{\BaseNTok}[1]{\textcolor[rgb]{0.00,0.00,0.81}{#1}}
\newcommand{\FloatTok}[1]{\textcolor[rgb]{0.00,0.00,0.81}{#1}}
\newcommand{\ConstantTok}[1]{\textcolor[rgb]{0.00,0.00,0.00}{#1}}
\newcommand{\CharTok}[1]{\textcolor[rgb]{0.31,0.60,0.02}{#1}}
\newcommand{\SpecialCharTok}[1]{\textcolor[rgb]{0.00,0.00,0.00}{#1}}
\newcommand{\StringTok}[1]{\textcolor[rgb]{0.31,0.60,0.02}{#1}}
\newcommand{\VerbatimStringTok}[1]{\textcolor[rgb]{0.31,0.60,0.02}{#1}}
\newcommand{\SpecialStringTok}[1]{\textcolor[rgb]{0.31,0.60,0.02}{#1}}
\newcommand{\ImportTok}[1]{#1}
\newcommand{\CommentTok}[1]{\textcolor[rgb]{0.56,0.35,0.01}{\textit{#1}}}
\newcommand{\DocumentationTok}[1]{\textcolor[rgb]{0.56,0.35,0.01}{\textbf{\textit{#1}}}}
\newcommand{\AnnotationTok}[1]{\textcolor[rgb]{0.56,0.35,0.01}{\textbf{\textit{#1}}}}
\newcommand{\CommentVarTok}[1]{\textcolor[rgb]{0.56,0.35,0.01}{\textbf{\textit{#1}}}}
\newcommand{\OtherTok}[1]{\textcolor[rgb]{0.56,0.35,0.01}{#1}}
\newcommand{\FunctionTok}[1]{\textcolor[rgb]{0.00,0.00,0.00}{#1}}
\newcommand{\VariableTok}[1]{\textcolor[rgb]{0.00,0.00,0.00}{#1}}
\newcommand{\ControlFlowTok}[1]{\textcolor[rgb]{0.13,0.29,0.53}{\textbf{#1}}}
\newcommand{\OperatorTok}[1]{\textcolor[rgb]{0.81,0.36,0.00}{\textbf{#1}}}
\newcommand{\BuiltInTok}[1]{#1}
\newcommand{\ExtensionTok}[1]{#1}
\newcommand{\PreprocessorTok}[1]{\textcolor[rgb]{0.56,0.35,0.01}{\textit{#1}}}
\newcommand{\AttributeTok}[1]{\textcolor[rgb]{0.77,0.63,0.00}{#1}}
\newcommand{\RegionMarkerTok}[1]{#1}
\newcommand{\InformationTok}[1]{\textcolor[rgb]{0.56,0.35,0.01}{\textbf{\textit{#1}}}}
\newcommand{\WarningTok}[1]{\textcolor[rgb]{0.56,0.35,0.01}{\textbf{\textit{#1}}}}
\newcommand{\AlertTok}[1]{\textcolor[rgb]{0.94,0.16,0.16}{#1}}
\newcommand{\ErrorTok}[1]{\textcolor[rgb]{0.64,0.00,0.00}{\textbf{#1}}}
\newcommand{\NormalTok}[1]{#1}
\usepackage{graphicx,grffile}
\makeatletter
\def\maxwidth{\ifdim\Gin@nat@width>\linewidth\linewidth\else\Gin@nat@width\fi}
\def\maxheight{\ifdim\Gin@nat@height>\textheight\textheight\else\Gin@nat@height\fi}
\makeatother
% Scale images if necessary, so that they will not overflow the page
% margins by default, and it is still possible to overwrite the defaults
% using explicit options in \includegraphics[width, height, ...]{}
\setkeys{Gin}{width=\maxwidth,height=\maxheight,keepaspectratio}
\IfFileExists{parskip.sty}{%
\usepackage{parskip}
}{% else
\setlength{\parindent}{0pt}
\setlength{\parskip}{6pt plus 2pt minus 1pt}
}
\setlength{\emergencystretch}{3em}  % prevent overfull lines
\providecommand{\tightlist}{%
  \setlength{\itemsep}{0pt}\setlength{\parskip}{0pt}}
\setcounter{secnumdepth}{0}
% Redefines (sub)paragraphs to behave more like sections
\ifx\paragraph\undefined\else
\let\oldparagraph\paragraph
\renewcommand{\paragraph}[1]{\oldparagraph{#1}\mbox{}}
\fi
\ifx\subparagraph\undefined\else
\let\oldsubparagraph\subparagraph
\renewcommand{\subparagraph}[1]{\oldsubparagraph{#1}\mbox{}}
\fi

% set default figure placement to htbp
\makeatletter
\def\fps@figure{htbp}
\makeatother

\usepackage{bbm}

\title{Homework - DSSC Core Course}
\author{Aleksandra Greshnova}
\date{}

\begin{document}
\maketitle

\subsection{Homework2}\label{homework2}

\begin{Shaded}
\begin{Highlighting}[]
\KeywordTok{setwd}\NormalTok{(}\StringTok{"C:/Users/sasha"}\NormalTok{)}

\CommentTok{#read data}
\NormalTok{data <-}\StringTok{ }\KeywordTok{read.delim}\NormalTok{(}\StringTok{'trace1.txt'}\NormalTok{)}

\CommentTok{#create dataframe with two columns - value and time}
\NormalTok{data}\OperatorTok{$}\NormalTok{time <-}\StringTok{ }\KeywordTok{seq}\NormalTok{(}\DataTypeTok{from =} \DecValTok{0}\NormalTok{, }\DataTypeTok{to =} \DecValTok{100}\NormalTok{, }\DataTypeTok{length.out =} \DecValTok{1999999}\NormalTok{)}
\KeywordTok{names}\NormalTok{(data) <-}\StringTok{ }\KeywordTok{c}\NormalTok{(}\StringTok{'value'}\NormalTok{, }\StringTok{'time'}\NormalTok{)}

\CommentTok{#upload libraries}
\KeywordTok{library}\NormalTok{(dplyr)}
\end{Highlighting}
\end{Shaded}

\begin{verbatim}
## 
## Attaching package: 'dplyr'
\end{verbatim}

\begin{verbatim}
## The following objects are masked from 'package:stats':
## 
##     filter, lag
\end{verbatim}

\begin{verbatim}
## The following objects are masked from 'package:base':
## 
##     intersect, setdiff, setequal, union
\end{verbatim}

\begin{Shaded}
\begin{Highlighting}[]
\KeywordTok{library}\NormalTok{(ggplot2)}
\KeywordTok{library}\NormalTok{(scales)}
\KeywordTok{library}\NormalTok{(zoo)}
\end{Highlighting}
\end{Shaded}

\begin{verbatim}
## 
## Attaching package: 'zoo'
\end{verbatim}

\begin{verbatim}
## The following objects are masked from 'package:base':
## 
##     as.Date, as.Date.numeric
\end{verbatim}

\begin{Shaded}
\begin{Highlighting}[]
\CommentTok{#create dataframe with intervals}
\NormalTok{interval_point <-}\StringTok{ }\KeywordTok{seq}\NormalTok{(}\DataTypeTok{from =} \KeywordTok{min}\NormalTok{(data}\OperatorTok{$}\NormalTok{value), }\DataTypeTok{to =} \KeywordTok{max}\NormalTok{(data}\OperatorTok{$}\NormalTok{value), }
    \DataTypeTok{by =}\NormalTok{ (}\KeywordTok{max}\NormalTok{(data}\OperatorTok{$}\NormalTok{value)}\OperatorTok{-}\KeywordTok{min}\NormalTok{(data}\OperatorTok{$}\NormalTok{value))}\OperatorTok{/}\DecValTok{100}\NormalTok{)}

\NormalTok{intervals <-}\StringTok{ }\KeywordTok{as.data.frame}\NormalTok{(}\KeywordTok{matrix}\NormalTok{(}\DecValTok{0}\NormalTok{, }\DataTypeTok{ncol =} \DecValTok{2}\NormalTok{, }\DataTypeTok{nrow =} \DecValTok{100}\NormalTok{))}

\ControlFlowTok{for}\NormalTok{ (i }\ControlFlowTok{in}\NormalTok{ (}\DecValTok{1}\OperatorTok{:}\NormalTok{(}\KeywordTok{length}\NormalTok{(interval_point)}\OperatorTok{-}\DecValTok{1}\NormalTok{)))\{}
\NormalTok{  intervals[i,}\DecValTok{1}\NormalTok{] <-}\StringTok{ }\NormalTok{interval_point[i]}
\NormalTok{  intervals[i,}\DecValTok{2}\NormalTok{] <-}\StringTok{ }\NormalTok{interval_point[i}\OperatorTok{+}\DecValTok{1}\NormalTok{]}
\NormalTok{\}}

\KeywordTok{colnames}\NormalTok{(intervals) <-}\StringTok{ }\KeywordTok{c}\NormalTok{(}\StringTok{'lower'}\NormalTok{, }\StringTok{'upper'}\NormalTok{)}

\KeywordTok{head}\NormalTok{(intervals)}
\end{Highlighting}
\end{Shaded}

\begin{verbatim}
##       lower     upper
## 1 -130.3542 -127.8336
## 2 -127.8336 -125.3130
## 3 -125.3130 -122.7924
## 4 -122.7924 -120.2718
## 5 -120.2718 -117.7512
## 6 -117.7512 -115.2306
\end{verbatim}

\begin{Shaded}
\begin{Highlighting}[]
\CommentTok{#create dataframe for histogram}
\NormalTok{data_cut <-}\StringTok{ }\KeywordTok{cut}\NormalTok{(data}\OperatorTok{$}\NormalTok{value, interval_point)}
\NormalTok{data_freq <-}\StringTok{ }\KeywordTok{data.frame}\NormalTok{(}\KeywordTok{table}\NormalTok{(data_cut))}
\NormalTok{data_freq <-}\StringTok{ }\KeywordTok{select}\NormalTok{(data_freq, }\OperatorTok{-}\NormalTok{data_cut)}
\NormalTok{data_freq <-}\StringTok{ }\KeywordTok{cbind}\NormalTok{(data_freq, intervals)}
\NormalTok{data_freq}\OperatorTok{$}\NormalTok{norm_freq <-}\StringTok{ }\NormalTok{data_freq}\OperatorTok{$}\NormalTok{Freq}\OperatorTok{/}\NormalTok{(}\KeywordTok{sum}\NormalTok{(data_freq}\OperatorTok{$}\NormalTok{Freq)}\OperatorTok{*}\NormalTok{(data_freq}\OperatorTok{$}\NormalTok{upper}\OperatorTok{-}\NormalTok{data_freq}\OperatorTok{$}\NormalTok{lower))}

\CommentTok{#function for generation of samples and calculating their freqs}
\NormalTok{sample_freq <-}\StringTok{ }\ControlFlowTok{function}\NormalTok{(df)\{}
\NormalTok{  sample <-}\StringTok{ }\KeywordTok{sample}\NormalTok{(df}\OperatorTok{$}\NormalTok{value, }\KeywordTok{length}\NormalTok{(df}\OperatorTok{$}\NormalTok{value)}\OperatorTok{/}\DecValTok{2}\NormalTok{)}
\NormalTok{  sample_cut <-}\StringTok{ }\KeywordTok{cut}\NormalTok{(sample, interval_point)}
\NormalTok{  sample_freq <-}\StringTok{ }\KeywordTok{data.frame}\NormalTok{(}\KeywordTok{table}\NormalTok{(sample_cut))}
\NormalTok{  sample_freq}\OperatorTok{$}\NormalTok{norm_freq <-}\StringTok{ }\NormalTok{sample_freq}\OperatorTok{$}\NormalTok{Freq}\OperatorTok{/}\NormalTok{(}\KeywordTok{sum}\NormalTok{(sample_freq}\OperatorTok{$}\NormalTok{Freq)}\OperatorTok{*}\NormalTok{(data_freq}\OperatorTok{$}\NormalTok{upper}\OperatorTok{-}\NormalTok{data_freq}\OperatorTok{$}\NormalTok{lower))}
  \KeywordTok{return}\NormalTok{(sample_freq}\OperatorTok{$}\NormalTok{norm_freq)}
\NormalTok{\}}

\CommentTok{#calculating upper and lower boarder of conf int}
\NormalTok{samples <-}\StringTok{ }\KeywordTok{replicate}\NormalTok{(}\DataTypeTok{n =} \DecValTok{100}\NormalTok{, }\KeywordTok{sample_freq}\NormalTok{(data), }\DataTypeTok{simplify =} \OtherTok{FALSE}\NormalTok{ )}
\NormalTok{samples_df <-}\StringTok{ }\KeywordTok{as.data.frame}\NormalTok{(}\KeywordTok{do.call}\NormalTok{(cbind, samples))}
\NormalTok{data_freq <-}\StringTok{ }\KeywordTok{cbind}\NormalTok{(data_freq, }\KeywordTok{t}\NormalTok{(}\KeywordTok{as.data.frame}\NormalTok{(}\KeywordTok{apply}\NormalTok{(samples_df, }
                                                    \DecValTok{1}\NormalTok{, }\ControlFlowTok{function}\NormalTok{(x) }\KeywordTok{quantile}\NormalTok{(x, }\KeywordTok{c}\NormalTok{(.}\DecValTok{025}\NormalTok{, .}\DecValTok{975}\NormalTok{))))))}
\NormalTok{data_freq}\OperatorTok{$}\NormalTok{middle <-}\StringTok{ }\NormalTok{data_freq}\OperatorTok{$}\NormalTok{lower }\OperatorTok{+}\StringTok{ }\NormalTok{(data_freq}\OperatorTok{$}\NormalTok{upper }\OperatorTok{-}\StringTok{ }\NormalTok{data_freq}\OperatorTok{$}\NormalTok{lower)}\OperatorTok{/}\DecValTok{2}

\CommentTok{#subsetting data for graph}
\NormalTok{data_for_graph <-}\StringTok{ }\KeywordTok{select}\NormalTok{(data_freq, }\OperatorTok{-}\KeywordTok{c}\NormalTok{(}\StringTok{'Freq'}\NormalTok{, }\StringTok{'lower'}\NormalTok{, }\StringTok{'upper'}\NormalTok{))}
\KeywordTok{colnames}\NormalTok{(data_for_graph) <-}\StringTok{ }\KeywordTok{c}\NormalTok{(}\StringTok{'norm_freq'}\NormalTok{, }\StringTok{'lower'}\NormalTok{, }\StringTok{'upper'}\NormalTok{, }\StringTok{'middle'}\NormalTok{)}
\KeywordTok{head}\NormalTok{(data_for_graph)}
\end{Highlighting}
\end{Shaded}

\begin{verbatim}
##       norm_freq        lower        upper    middle
## V1 0.000000e+00 0.000000e+00 0.000000e+00 -129.0939
## V2 0.000000e+00 0.000000e+00 0.000000e+00 -126.5733
## V3 1.983658e-07 0.000000e+00 3.967320e-07 -124.0527
## V4 1.983658e-07 0.000000e+00 3.967320e-07 -121.5321
## V5 9.918291e-07 3.967316e-07 1.795211e-06 -119.0115
## V6 5.950975e-07 0.000000e+00 1.190196e-06 -116.4909
\end{verbatim}

\begin{Shaded}
\begin{Highlighting}[]
\CommentTok{#creating a plot}
\NormalTok{plot <-}\StringTok{ }\KeywordTok{ggplot}\NormalTok{()}\OperatorTok{+}
\StringTok{  }\KeywordTok{geom_line}\NormalTok{(}\DataTypeTok{data=}\NormalTok{data_for_graph, }\KeywordTok{aes}\NormalTok{(}\DataTypeTok{x=}\NormalTok{middle, }\DataTypeTok{y=}\NormalTok{norm_freq))}\OperatorTok{+}
\StringTok{  }\KeywordTok{geom_point}\NormalTok{(}\DataTypeTok{data=}\NormalTok{data_for_graph, }\KeywordTok{aes}\NormalTok{(}\DataTypeTok{x=}\NormalTok{middle, }\DataTypeTok{y=}\NormalTok{norm_freq), }\DataTypeTok{size=}\DecValTok{1}\NormalTok{)}\OperatorTok{+}
\StringTok{  }\KeywordTok{geom_errorbar}\NormalTok{(}\DataTypeTok{data=}\NormalTok{data_for_graph, }\KeywordTok{aes}\NormalTok{(}\DataTypeTok{x=}\NormalTok{middle, }\DataTypeTok{ymin=}\NormalTok{lower, }\DataTypeTok{ymax=}\NormalTok{upper), }\DataTypeTok{color=}\StringTok{'red'}\NormalTok{)}\OperatorTok{+}
\StringTok{  }\KeywordTok{scale_y_log10}\NormalTok{(}\DataTypeTok{breaks =} \KeywordTok{trans_breaks}\NormalTok{(}\StringTok{"log10"}\NormalTok{, }\ControlFlowTok{function}\NormalTok{(x) }\DecValTok{10}\OperatorTok{^}\NormalTok{x),}
                \DataTypeTok{labels =} \KeywordTok{trans_format}\NormalTok{(}\StringTok{"log10"}\NormalTok{, }\KeywordTok{math_format}\NormalTok{(}\DecValTok{10}\OperatorTok{^}\NormalTok{.x)))}\OperatorTok{+}
\StringTok{  }\KeywordTok{xlab}\NormalTok{(}\StringTok{"Voltage [microV]"}\NormalTok{)}\OperatorTok{+}
\StringTok{  }\KeywordTok{ylab}\NormalTok{(}\StringTok{"P(V)"}\NormalTok{)}\OperatorTok{+}
\StringTok{  }\KeywordTok{ggtitle}\NormalTok{(}\StringTok{'PDF'}\NormalTok{)}\OperatorTok{+}
\StringTok{  }\KeywordTok{theme}\NormalTok{(}\DataTypeTok{plot.title =} \KeywordTok{element_text}\NormalTok{(}\DataTypeTok{hjust =} \FloatTok{0.5}\NormalTok{))}

\NormalTok{plot}
\end{Highlighting}
\end{Shaded}

\begin{verbatim}
## Warning: Transformation introduced infinite values in continuous y-axis

## Warning: Transformation introduced infinite values in continuous y-axis

## Warning: Transformation introduced infinite values in continuous y-axis

## Warning: Transformation introduced infinite values in continuous y-axis
\end{verbatim}

\includegraphics{Homework_files/figure-latex/unnamed-chunk-1-1.pdf}

\begin{Shaded}
\begin{Highlighting}[]
\CommentTok{#subset 50000 entries}
\NormalTok{data_2_}\DecValTok{5}\NormalTok{ <-}\StringTok{ }\NormalTok{data[}\DecValTok{1}\OperatorTok{:}\DecValTok{50000}\NormalTok{,]}

\CommentTok{#plot a subset (voltage vs time)}
\NormalTok{plot_data_2_}\DecValTok{5}\NormalTok{ <-}\StringTok{ }\KeywordTok{ggplot}\NormalTok{(data_2_}\DecValTok{5}\NormalTok{, }\KeywordTok{aes}\NormalTok{(}\DataTypeTok{x=}\NormalTok{time, }\DataTypeTok{y=}\NormalTok{value))}\OperatorTok{+}
\StringTok{  }\KeywordTok{geom_line}\NormalTok{(}\DataTypeTok{color =} \StringTok{"blue"}\NormalTok{)}\OperatorTok{+}
\StringTok{  }\KeywordTok{geom_hline}\NormalTok{(}\DataTypeTok{yintercept=}\KeywordTok{c}\NormalTok{(}\OperatorTok{-}\DecValTok{30}\NormalTok{,}\OperatorTok{-}\DecValTok{50}\NormalTok{), }\DataTypeTok{linetype=}\StringTok{"dashed"}\NormalTok{, }\DataTypeTok{color =} \StringTok{"red"}\NormalTok{)}\OperatorTok{+}
\StringTok{  }\KeywordTok{xlab}\NormalTok{(}\StringTok{"Time [s]"}\NormalTok{)}\OperatorTok{+}
\StringTok{  }\KeywordTok{ylab}\NormalTok{(}\StringTok{"Voltage [microV]"}\NormalTok{)}\OperatorTok{+}
\StringTok{  }\KeywordTok{ggtitle}\NormalTok{(}\StringTok{"A 2.5s example trace - original data"}\NormalTok{)}\OperatorTok{+}
\StringTok{  }\KeywordTok{theme}\NormalTok{(}\DataTypeTok{plot.title =} \KeywordTok{element_text}\NormalTok{(}\DataTypeTok{hjust =} \FloatTok{0.5}\NormalTok{))}

\NormalTok{plot_data_2_}\DecValTok{5}
\end{Highlighting}
\end{Shaded}

\includegraphics{Homework_files/figure-latex/unnamed-chunk-1-2.pdf}

\begin{Shaded}
\begin{Highlighting}[]
\CommentTok{#function which counts number of peaks under a certain threshold}
\NormalTok{peaks_count <-}\StringTok{ }\ControlFlowTok{function}\NormalTok{(d, t)\{}
  
\NormalTok{  d}\OperatorTok{$}\NormalTok{id <-}\StringTok{ }\KeywordTok{seq.int}\NormalTok{(}\KeywordTok{nrow}\NormalTok{(d))}
  
\NormalTok{  d}\OperatorTok{$}\NormalTok{difference <-}\StringTok{ }\KeywordTok{c}\NormalTok{(}\DecValTok{0}\NormalTok{, }\KeywordTok{diff}\NormalTok{(d}\OperatorTok{$}\NormalTok{value))}
  
\NormalTok{  indexes <-}\StringTok{ }\KeywordTok{which}\NormalTok{(d}\OperatorTok{$}\NormalTok{difference }\OperatorTok{<}\StringTok{ }\DecValTok{0}\NormalTok{)}
  
\NormalTok{  d_descending <-}\StringTok{ }\NormalTok{d[indexes, ]}
  
\NormalTok{  d_descending_under_t <-}\StringTok{ }\NormalTok{d_descending[}\KeywordTok{which}\NormalTok{(d_descending}\OperatorTok{$}\NormalTok{value }\OperatorTok{<}\StringTok{ }\NormalTok{t),]}
  
\NormalTok{  d_descending_under_t}\OperatorTok{$}\NormalTok{diff_id <-}\StringTok{ }\KeywordTok{c}\NormalTok{(}\KeywordTok{diff}\NormalTok{(d_descending_under_t}\OperatorTok{$}\NormalTok{id), }\DecValTok{0}\NormalTok{)}
  
\NormalTok{  d_peaks <-}\StringTok{ }\NormalTok{d_descending_under_t[}\KeywordTok{which}\NormalTok{(d_descending_under_t}\OperatorTok{$}\NormalTok{diff_id }\OperatorTok{>}\StringTok{ }\DecValTok{1}\NormalTok{), ]}
  
  \KeywordTok{return}\NormalTok{(}\KeywordTok{length}\NormalTok{(d_peaks}\OperatorTok{$}\NormalTok{value))}
\NormalTok{\}}

\CommentTok{#scan a range of thresholds}
\NormalTok{spikes_data <-}\StringTok{ }\KeywordTok{data.frame}\NormalTok{(}\KeywordTok{matrix}\NormalTok{(}\DataTypeTok{ncol =} \DecValTok{2}\NormalTok{, }\DataTypeTok{nrow =} \DecValTok{0}\NormalTok{))}

\ControlFlowTok{for}\NormalTok{(i }\ControlFlowTok{in} \OperatorTok{-}\DecValTok{30}\OperatorTok{:-}\DecValTok{70}\NormalTok{)\{}
\NormalTok{  spikes_data <-}\StringTok{ }\KeywordTok{rbind}\NormalTok{(spikes_data, }\KeywordTok{c}\NormalTok{(i, }\KeywordTok{peaks_count}\NormalTok{(}\DataTypeTok{d=}\NormalTok{data_2_}\DecValTok{5}\NormalTok{, }\DataTypeTok{t=}\NormalTok{i)))}
\NormalTok{\}}

\KeywordTok{colnames}\NormalTok{(spikes_data) <-}\StringTok{ }\KeywordTok{c}\NormalTok{(}\StringTok{'threshold'}\NormalTok{, }\StringTok{'N of peaks'}\NormalTok{)}
\KeywordTok{plot}\NormalTok{(spikes_data)}
\end{Highlighting}
\end{Shaded}

\includegraphics{Homework_files/figure-latex/unnamed-chunk-1-3.pdf}

\begin{Shaded}
\begin{Highlighting}[]
\CommentTok{#smooting_data}
\NormalTok{data_smooth <-}\StringTok{ }\NormalTok{data[}\DecValTok{1}\OperatorTok{:}\DecValTok{1999900}\NormalTok{,]}
\NormalTok{data_smooth}\OperatorTok{$}\NormalTok{value <-}\StringTok{ }\NormalTok{data_smooth}\OperatorTok{$}\NormalTok{value }\OperatorTok{-}\StringTok{ }\KeywordTok{rollapply}\NormalTok{(data}\OperatorTok{$}\NormalTok{value, }\DataTypeTok{width=}\DecValTok{100}\NormalTok{, }\DataTypeTok{FUN=}\NormalTok{mean)}
\NormalTok{data_smooth_2_}\DecValTok{5}\NormalTok{ <-}\StringTok{ }\NormalTok{data_smooth[}\DecValTok{1}\OperatorTok{:}\DecValTok{50000}\NormalTok{,]}

\CommentTok{#plot a subset - smoothed data (voltage vs time)}
\NormalTok{plot_data_smooth_2_}\DecValTok{5}\NormalTok{ <-}\StringTok{ }\KeywordTok{ggplot}\NormalTok{(data_smooth_2_}\DecValTok{5}\NormalTok{, }\KeywordTok{aes}\NormalTok{(}\DataTypeTok{x=}\NormalTok{time, }\DataTypeTok{y=}\NormalTok{value))}\OperatorTok{+}
\StringTok{  }\KeywordTok{geom_line}\NormalTok{(}\DataTypeTok{color =} \StringTok{"blue"}\NormalTok{)}\OperatorTok{+}
\StringTok{  }\KeywordTok{xlab}\NormalTok{(}\StringTok{"Time [s]"}\NormalTok{)}\OperatorTok{+}
\StringTok{  }\KeywordTok{ylab}\NormalTok{(}\StringTok{"Voltage [microV]"}\NormalTok{)}\OperatorTok{+}
\StringTok{  }\KeywordTok{ggtitle}\NormalTok{(}\StringTok{"A 2.5s example trace - smoothed data"}\NormalTok{)}\OperatorTok{+}
\StringTok{  }\KeywordTok{theme}\NormalTok{(}\DataTypeTok{plot.title =} \KeywordTok{element_text}\NormalTok{(}\DataTypeTok{hjust =} \FloatTok{0.5}\NormalTok{))}

\NormalTok{plot_data_smooth_2_}\DecValTok{5}
\end{Highlighting}
\end{Shaded}

\includegraphics{Homework_files/figure-latex/unnamed-chunk-1-4.pdf}

\begin{Shaded}
\begin{Highlighting}[]
\CommentTok{#counting peaks in smoothed data}
\NormalTok{spikes_sm_data <-}\StringTok{ }\KeywordTok{data.frame}\NormalTok{(}\KeywordTok{matrix}\NormalTok{(}\DataTypeTok{ncol =} \DecValTok{2}\NormalTok{, }\DataTypeTok{nrow =} \DecValTok{0}\NormalTok{))}

\ControlFlowTok{for}\NormalTok{(i }\ControlFlowTok{in} \OperatorTok{-}\DecValTok{30}\OperatorTok{:-}\DecValTok{70}\NormalTok{)\{}
\NormalTok{  spikes_sm_data <-}\StringTok{ }\KeywordTok{rbind}\NormalTok{(spikes_sm_data, }\KeywordTok{c}\NormalTok{(i, }\KeywordTok{peaks_count}\NormalTok{(}\DataTypeTok{d=}\NormalTok{data_smooth_2_}\DecValTok{5}\NormalTok{, }\DataTypeTok{t=}\NormalTok{i)))}
\NormalTok{\}}

\KeywordTok{colnames}\NormalTok{(spikes_sm_data) <-}\StringTok{ }\KeywordTok{c}\NormalTok{(}\StringTok{'threshold'}\NormalTok{, }\StringTok{'N of peaks'}\NormalTok{)}

\KeywordTok{plot}\NormalTok{(spikes_sm_data)}
\end{Highlighting}
\end{Shaded}

\includegraphics{Homework_files/figure-latex/unnamed-chunk-1-5.pdf}

\begin{Shaded}
\begin{Highlighting}[]
\CommentTok{#function which returns a coordinates of peaks}
\NormalTok{peaks_call <-}\StringTok{ }\ControlFlowTok{function}\NormalTok{(d, t)\{}
  
\NormalTok{  d}\OperatorTok{$}\NormalTok{id <-}\StringTok{ }\KeywordTok{seq.int}\NormalTok{(}\KeywordTok{nrow}\NormalTok{(d))}
  
\NormalTok{  d}\OperatorTok{$}\NormalTok{difference <-}\StringTok{ }\KeywordTok{c}\NormalTok{(}\DecValTok{0}\NormalTok{, }\KeywordTok{diff}\NormalTok{(d}\OperatorTok{$}\NormalTok{value))}
  
\NormalTok{  indexes <-}\StringTok{ }\KeywordTok{which}\NormalTok{(d}\OperatorTok{$}\NormalTok{difference }\OperatorTok{<}\StringTok{ }\DecValTok{0}\NormalTok{)}
  
\NormalTok{  d_descending <-}\StringTok{ }\NormalTok{d[indexes, ]}
  
\NormalTok{  d_descending_under_t <-}\StringTok{ }\NormalTok{d_descending[}\KeywordTok{which}\NormalTok{(d_descending}\OperatorTok{$}\NormalTok{value }\OperatorTok{<}\StringTok{ }\NormalTok{t),]}
  
\NormalTok{  d_descending_under_t}\OperatorTok{$}\NormalTok{diff_id <-}\StringTok{ }\KeywordTok{c}\NormalTok{(}\KeywordTok{diff}\NormalTok{(d_descending_under_t}\OperatorTok{$}\NormalTok{id), }\DecValTok{0}\NormalTok{)}
  
\NormalTok{  d_peaks <-}\StringTok{ }\NormalTok{d_descending_under_t[}\KeywordTok{which}\NormalTok{(d_descending_under_t}\OperatorTok{$}\NormalTok{diff_id }\OperatorTok{>}\StringTok{ }\DecValTok{1}\NormalTok{), ]}
  
  \KeywordTok{return}\NormalTok{(}\KeywordTok{data.frame}\NormalTok{(}\DataTypeTok{value =}\NormalTok{ d_peaks}\OperatorTok{$}\NormalTok{value, }\DataTypeTok{time =}\NormalTok{ d_peaks}\OperatorTok{$}\NormalTok{time))}
\NormalTok{\}}

\CommentTok{#peaks, threashold = -50 microV}
\NormalTok{peaks_sm_data_2_}\DecValTok{5}\NormalTok{ <-}\StringTok{ }\KeywordTok{peaks_call}\NormalTok{(}\DataTypeTok{d=}\NormalTok{data_smooth_2_}\DecValTok{5}\NormalTok{, }\DataTypeTok{t=}\OperatorTok{-}\DecValTok{50}\NormalTok{)}
\NormalTok{peaks_data_2_}\DecValTok{5}\NormalTok{ <-}\StringTok{ }\KeywordTok{peaks_call}\NormalTok{(}\DataTypeTok{d=}\NormalTok{data_2_}\DecValTok{5}\NormalTok{, }\DataTypeTok{t=}\OperatorTok{-}\DecValTok{50}\NormalTok{)}

\CommentTok{#plotting data with peaks}
\NormalTok{plot_data_2_5_peaks <-}\StringTok{ }\KeywordTok{ggplot}\NormalTok{(data_2_}\DecValTok{5}\NormalTok{, }\KeywordTok{aes}\NormalTok{(}\DataTypeTok{x=}\NormalTok{time, }\DataTypeTok{y=}\NormalTok{value))}\OperatorTok{+}
\StringTok{  }\KeywordTok{geom_line}\NormalTok{(}\DataTypeTok{color =} \StringTok{"blue"}\NormalTok{)}\OperatorTok{+}
\StringTok{  }\KeywordTok{geom_hline}\NormalTok{(}\DataTypeTok{yintercept=}\KeywordTok{c}\NormalTok{(}\OperatorTok{-}\DecValTok{50}\NormalTok{), }\DataTypeTok{linetype=}\StringTok{"dashed"}\NormalTok{, }\DataTypeTok{color =} \StringTok{"red"}\NormalTok{)}\OperatorTok{+}
\StringTok{  }\KeywordTok{xlab}\NormalTok{(}\StringTok{"Time [s]"}\NormalTok{)}\OperatorTok{+}
\StringTok{  }\KeywordTok{ylab}\NormalTok{(}\StringTok{"Voltage [microV]"}\NormalTok{)}\OperatorTok{+}
\StringTok{  }\KeywordTok{ggtitle}\NormalTok{(}\StringTok{"A 2.5s example trace - smoothed data"}\NormalTok{)}\OperatorTok{+}
\StringTok{  }\KeywordTok{theme}\NormalTok{(}\DataTypeTok{plot.title =} \KeywordTok{element_text}\NormalTok{(}\DataTypeTok{hjust =} \FloatTok{0.5}\NormalTok{))}\OperatorTok{+}
\StringTok{  }\KeywordTok{geom_point}\NormalTok{(}\DataTypeTok{data=}\NormalTok{peaks_data_2_}\DecValTok{5}\NormalTok{, }\KeywordTok{aes}\NormalTok{(}\DataTypeTok{x=}\NormalTok{time, }\DataTypeTok{y=}\NormalTok{value), }\DataTypeTok{color=}\StringTok{'red'}\NormalTok{)}

\NormalTok{plot_data_2_5_peaks}
\end{Highlighting}
\end{Shaded}

\includegraphics{Homework_files/figure-latex/unnamed-chunk-1-6.pdf}

\begin{Shaded}
\begin{Highlighting}[]
\NormalTok{plot_data_sm_2_5_peaks <-}\StringTok{ }\KeywordTok{ggplot}\NormalTok{(data_smooth_2_}\DecValTok{5}\NormalTok{, }\KeywordTok{aes}\NormalTok{(}\DataTypeTok{x=}\NormalTok{time, }\DataTypeTok{y=}\NormalTok{value))}\OperatorTok{+}
\StringTok{  }\KeywordTok{geom_line}\NormalTok{(}\DataTypeTok{color =} \StringTok{"blue"}\NormalTok{)}\OperatorTok{+}
\StringTok{  }\KeywordTok{geom_hline}\NormalTok{(}\DataTypeTok{yintercept=}\KeywordTok{c}\NormalTok{(}\OperatorTok{-}\DecValTok{50}\NormalTok{), }\DataTypeTok{linetype=}\StringTok{"dashed"}\NormalTok{, }\DataTypeTok{color =} \StringTok{"red"}\NormalTok{)}\OperatorTok{+}
\StringTok{  }\KeywordTok{xlab}\NormalTok{(}\StringTok{"Time [s]"}\NormalTok{)}\OperatorTok{+}
\StringTok{  }\KeywordTok{ylab}\NormalTok{(}\StringTok{"Voltage [microV]"}\NormalTok{)}\OperatorTok{+}
\StringTok{  }\KeywordTok{ggtitle}\NormalTok{(}\StringTok{"A 2.5s example trace - smoothed data"}\NormalTok{)}\OperatorTok{+}
\StringTok{  }\KeywordTok{theme}\NormalTok{(}\DataTypeTok{plot.title =} \KeywordTok{element_text}\NormalTok{(}\DataTypeTok{hjust =} \FloatTok{0.5}\NormalTok{))}\OperatorTok{+}
\StringTok{  }\KeywordTok{geom_point}\NormalTok{(}\DataTypeTok{data=}\NormalTok{peaks_sm_data_2_}\DecValTok{5}\NormalTok{, }\KeywordTok{aes}\NormalTok{(}\DataTypeTok{x=}\NormalTok{time, }\DataTypeTok{y=}\NormalTok{value), }\DataTypeTok{color=}\StringTok{'red'}\NormalTok{)}

\NormalTok{plot_data_sm_2_5_peaks}
\end{Highlighting}
\end{Shaded}

\includegraphics{Homework_files/figure-latex/unnamed-chunk-1-7.pdf}

\begin{Shaded}
\begin{Highlighting}[]
\CommentTok{#counting_peaks}

\CommentTok{#function which returns an original df with peaks indicated}
\NormalTok{peaks_call_original <-}\StringTok{ }\ControlFlowTok{function}\NormalTok{(d, t)\{}
  
\NormalTok{  d}\OperatorTok{$}\NormalTok{id <-}\StringTok{ }\KeywordTok{seq.int}\NormalTok{(}\KeywordTok{nrow}\NormalTok{(d))}
  
\NormalTok{  d}\OperatorTok{$}\NormalTok{difference <-}\StringTok{ }\KeywordTok{c}\NormalTok{(}\DecValTok{0}\NormalTok{, }\KeywordTok{diff}\NormalTok{(d}\OperatorTok{$}\NormalTok{value))}
  
\NormalTok{  indexes <-}\StringTok{ }\KeywordTok{which}\NormalTok{(d}\OperatorTok{$}\NormalTok{difference }\OperatorTok{<}\StringTok{ }\DecValTok{0}\NormalTok{)}
  
\NormalTok{  d_descending <-}\StringTok{ }\NormalTok{d[indexes, ]}
  
\NormalTok{  d_descending_under_t <-}\StringTok{ }\NormalTok{d_descending[}\KeywordTok{which}\NormalTok{(d_descending}\OperatorTok{$}\NormalTok{value }\OperatorTok{<}\StringTok{ }\NormalTok{t),]}
  
\NormalTok{  d_descending_under_t}\OperatorTok{$}\NormalTok{diff_id <-}\StringTok{ }\KeywordTok{c}\NormalTok{(}\KeywordTok{diff}\NormalTok{(d_descending_under_t}\OperatorTok{$}\NormalTok{id), }\DecValTok{0}\NormalTok{)}
  
\NormalTok{  d_peaks <-}\StringTok{ }\NormalTok{d_descending_under_t[}\KeywordTok{which}\NormalTok{(d_descending_under_t}\OperatorTok{$}\NormalTok{diff_id }\OperatorTok{>}\StringTok{ }\DecValTok{1}\NormalTok{), ]}
  
\NormalTok{  only_peaks <-}\StringTok{ }\KeywordTok{data.frame}\NormalTok{(}\DataTypeTok{value =}\NormalTok{ d_peaks}\OperatorTok{$}\NormalTok{value, }\DataTypeTok{time =}\NormalTok{ d_peaks}\OperatorTok{$}\NormalTok{time)}
  
\NormalTok{  spikes <-}\StringTok{ }\KeywordTok{merge.data.frame}\NormalTok{(only_peaks, data_smooth_2_}\DecValTok{5}\NormalTok{, }\DataTypeTok{by=}\StringTok{'time'}\NormalTok{, }\DataTypeTok{all.x =}\NormalTok{ T, }\DataTypeTok{all.y =}\NormalTok{ T)}
  
  \KeywordTok{colnames}\NormalTok{(spikes) <-}\StringTok{ }\KeywordTok{c}\NormalTok{(}\StringTok{'time'}\NormalTok{, }\StringTok{'peaks'}\NormalTok{, }\StringTok{'signal'}\NormalTok{)}
  
  \KeywordTok{return}\NormalTok{(spikes)}
\NormalTok{\}}

\NormalTok{spikes_in_original_data <-}\StringTok{ }\KeywordTok{peaks_call_original}\NormalTok{(data_smooth_2_}\DecValTok{5}\NormalTok{, }\OperatorTok{-}\DecValTok{50}\NormalTok{)}
\KeywordTok{colnames}\NormalTok{(spikes_in_original_data) <-}\StringTok{ }\KeywordTok{c}\NormalTok{(}\StringTok{'time'}\NormalTok{, }\StringTok{'true_spikes'}\NormalTok{, }\StringTok{'original_signal'}\NormalTok{)}
\KeywordTok{head}\NormalTok{(spikes_in_original_data)}
\end{Highlighting}
\end{Shaded}

\begin{verbatim}
##           time true_spikes original_signal
## 1 0.000000e+00          NA        3.411508
## 2 5.000005e-05          NA        6.645876
## 3 1.000001e-04          NA        1.483808
## 4 1.500002e-04          NA        1.611974
## 5 2.000002e-04          NA        4.563960
## 6 2.500003e-04          NA        5.100590
\end{verbatim}

\begin{Shaded}
\begin{Highlighting}[]
\CommentTok{#function which retuen a df with coordinates of peaks}
\NormalTok{peaks_call <-}\StringTok{ }\ControlFlowTok{function}\NormalTok{(d, t)\{}
  
\NormalTok{  d}\OperatorTok{$}\NormalTok{id <-}\StringTok{ }\KeywordTok{seq.int}\NormalTok{(}\KeywordTok{nrow}\NormalTok{(d))}
  
\NormalTok{  d}\OperatorTok{$}\NormalTok{difference <-}\StringTok{ }\KeywordTok{c}\NormalTok{(}\DecValTok{0}\NormalTok{, }\KeywordTok{diff}\NormalTok{(d}\OperatorTok{$}\NormalTok{value))}
  
\NormalTok{  indexes <-}\StringTok{ }\KeywordTok{which}\NormalTok{(d}\OperatorTok{$}\NormalTok{difference }\OperatorTok{<}\StringTok{ }\DecValTok{0}\NormalTok{)}
  
\NormalTok{  d_descending <-}\StringTok{ }\NormalTok{d[indexes, ]}
  
\NormalTok{  d_descending_under_t <-}\StringTok{ }\NormalTok{d_descending[}\KeywordTok{which}\NormalTok{(d_descending}\OperatorTok{$}\NormalTok{value }\OperatorTok{<}\StringTok{ }\NormalTok{t),]}
  
\NormalTok{  d_descending_under_t}\OperatorTok{$}\NormalTok{diff_id <-}\StringTok{ }\KeywordTok{c}\NormalTok{(}\KeywordTok{diff}\NormalTok{(d_descending_under_t}\OperatorTok{$}\NormalTok{id), }\DecValTok{0}\NormalTok{)}
  
\NormalTok{  d_peaks <-}\StringTok{ }\NormalTok{d_descending_under_t[}\KeywordTok{which}\NormalTok{(d_descending_under_t}\OperatorTok{$}\NormalTok{diff_id }\OperatorTok{>}\StringTok{ }\DecValTok{1}\NormalTok{), ]}
  
  \KeywordTok{return}\NormalTok{(}\KeywordTok{data.frame}\NormalTok{(}\DataTypeTok{value =}\NormalTok{ d_peaks}\OperatorTok{$}\NormalTok{value, }\DataTypeTok{time =}\NormalTok{ d_peaks}\OperatorTok{$}\NormalTok{time))}
\NormalTok{\}}

\NormalTok{peaks <-}\StringTok{ }\KeywordTok{peaks_call}\NormalTok{(data_2_}\DecValTok{5}\NormalTok{, }\OperatorTok{-}\DecValTok{50}\NormalTok{)}

\CommentTok{#creating df for ROC curve}
\NormalTok{data_for_roc <-}\StringTok{ }\ControlFlowTok{function}\NormalTok{(d_true, d_sample)\{}
  
\NormalTok{  result <-}\StringTok{ }\KeywordTok{merge.data.frame}\NormalTok{(d_true, d_sample, }\DataTypeTok{by=}\StringTok{'time'}\NormalTok{, }\DataTypeTok{all.x=}\NormalTok{T, }\DataTypeTok{all.y=}\NormalTok{T)}
  
\NormalTok{  results_peaks_subset <-}\StringTok{ }\NormalTok{result[}\OperatorTok{-}\KeywordTok{which}\NormalTok{(}\KeywordTok{is.na}\NormalTok{(result}\OperatorTok{$}\NormalTok{true_spikes)), ]}
  
\NormalTok{  TP <-}\StringTok{ }\KeywordTok{nrow}\NormalTok{(results_peaks_subset[}\KeywordTok{which}\NormalTok{(}\OperatorTok{!}\KeywordTok{is.na}\NormalTok{(results_peaks_subset}\OperatorTok{$}\NormalTok{value)), ])}
  
\NormalTok{  FN <-}\StringTok{ }\KeywordTok{nrow}\NormalTok{(results_peaks_subset) }\OperatorTok{-}\StringTok{ }\NormalTok{TP}
  
\NormalTok{  results_no_peaks_subset <-}\StringTok{ }\NormalTok{result[}\KeywordTok{which}\NormalTok{(}\KeywordTok{is.na}\NormalTok{(result}\OperatorTok{$}\NormalTok{true_spikes)), ]}
  
\NormalTok{  FP <-}\StringTok{ }\KeywordTok{nrow}\NormalTok{(results_no_peaks_subset[}\KeywordTok{which}\NormalTok{(}\OperatorTok{!}\KeywordTok{is.na}\NormalTok{(results_no_peaks_subset}\OperatorTok{$}\NormalTok{value)), ])}
\NormalTok{  TN <-}\StringTok{ }\KeywordTok{nrow}\NormalTok{(results_no_peaks_subset) }\OperatorTok{-}\StringTok{ }\NormalTok{FP}
  
  \KeywordTok{return}\NormalTok{(}\KeywordTok{c}\NormalTok{(TP, TN, FP, FN))}
  
\NormalTok{\}}

\NormalTok{ROC_data <-}\StringTok{ }\KeywordTok{data.frame}\NormalTok{(}\KeywordTok{matrix}\NormalTok{(}\DataTypeTok{ncol =} \DecValTok{5}\NormalTok{, }\DataTypeTok{nrow =} \DecValTok{0}\NormalTok{))}

\ControlFlowTok{for}\NormalTok{(i }\ControlFlowTok{in} \KeywordTok{seq}\NormalTok{(}\OperatorTok{-}\DecValTok{30}\NormalTok{,}\OperatorTok{-}\DecValTok{70}\NormalTok{,}\OperatorTok{-}\DecValTok{1}\NormalTok{))\{}
  
\NormalTok{  sample_peaks <-}\StringTok{ }\KeywordTok{peaks_call}\NormalTok{(data_2_}\DecValTok{5}\NormalTok{, }\DataTypeTok{t=}\NormalTok{i)}
\NormalTok{  peaks <-}\StringTok{ }\KeywordTok{data_for_roc}\NormalTok{(}\DataTypeTok{d_true=}\NormalTok{spikes_in_original_data, }\DataTypeTok{d_sample=}\NormalTok{sample_peaks)}
  
\NormalTok{  ROC_data <-}\StringTok{ }\KeywordTok{rbind}\NormalTok{(ROC_data, }\KeywordTok{c}\NormalTok{(peaks, i))}
\NormalTok{\}}

\KeywordTok{colnames}\NormalTok{(ROC_data) <-}\StringTok{ }\KeywordTok{c}\NormalTok{(}\StringTok{'TP'}\NormalTok{, }\StringTok{'TN'}\NormalTok{, }\StringTok{'FP'}\NormalTok{, }\StringTok{'FN'}\NormalTok{, }\StringTok{'threshold'}\NormalTok{)}

\KeywordTok{head}\NormalTok{(ROC_data)}
\end{Highlighting}
\end{Shaded}

\begin{verbatim}
##   TP    TN  FP FN threshold
## 1 18 49431 546  5       -30
## 2 18 49527 450  5       -31
## 3 18 49613 364  5       -32
## 4 18 49677 300  5       -33
## 5 18 49735 242  5       -34
## 6 18 49775 202  5       -35
\end{verbatim}

\begin{Shaded}
\begin{Highlighting}[]
\NormalTok{ROC_data}\OperatorTok{$}\NormalTok{TPR <-}\StringTok{ }\NormalTok{ROC_data}\OperatorTok{$}\NormalTok{TP}\OperatorTok{/}\NormalTok{(ROC_data}\OperatorTok{$}\NormalTok{TP}\OperatorTok{+}\NormalTok{ROC_data}\OperatorTok{$}\NormalTok{FN)}
\NormalTok{ROC_data}\OperatorTok{$}\NormalTok{FPR <-}\StringTok{ }\NormalTok{ROC_data}\OperatorTok{$}\NormalTok{FP}\OperatorTok{/}\NormalTok{(ROC_data}\OperatorTok{$}\NormalTok{FP}\OperatorTok{+}\NormalTok{ROC_data}\OperatorTok{$}\NormalTok{TN)}

\NormalTok{ROC_curve <-}\StringTok{ }\KeywordTok{ggplot}\NormalTok{(}\DataTypeTok{data=}\NormalTok{ROC_data, }\KeywordTok{aes}\NormalTok{(}\DataTypeTok{x=}\NormalTok{FPR, }\DataTypeTok{y=}\NormalTok{TPR))}\OperatorTok{+}
\StringTok{  }\KeywordTok{geom_point}\NormalTok{(}\DataTypeTok{color=}\StringTok{'red'}\NormalTok{)}\OperatorTok{+}
\StringTok{  }\KeywordTok{ggtitle}\NormalTok{(}\StringTok{'ROC curve'}\NormalTok{)}\OperatorTok{+}
\StringTok{  }\KeywordTok{theme}\NormalTok{(}\DataTypeTok{plot.title =} \KeywordTok{element_text}\NormalTok{(}\DataTypeTok{hjust =} \FloatTok{0.5}\NormalTok{))}

\NormalTok{ROC_curve}
\end{Highlighting}
\end{Shaded}

\includegraphics{Homework_files/figure-latex/unnamed-chunk-1-8.pdf}

\end{document}
